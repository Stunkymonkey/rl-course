\documentclass[a4paper]{article}
\usepackage[utf8]{inputenc}
\usepackage{textcomp}
\usepackage{geometry}
\geometry{ left=2cm, right=2cm, top=2cm, bottom=2cm, bindingoffset=5mm}
\usepackage{graphicx}
\usepackage{xcolor}
\usepackage{hyperref}
\date{}
\author{}
\usepackage{fancyhdr}
\pagestyle{fancy}
\fancyhf{}
\fancyhead[R]{Felix Bühler - 2973140\\ Jan Leusmann - 2893121\\  Jamie Ullerich - 3141241}
\fancyhead[L]{Reinforcement Learning \\ SS 2020}
\renewcommand{\headrulewidth}{0.5pt}
\usepackage{tikz}
\usetikzlibrary{calc}
\usepackage{amsmath}
\usepackage{cleveref}
\usepackage{subcaption}
\usepackage{array}
\usepackage{bbold}

\title{\textbf{Exercise 2}}

\begin{document}
\maketitle 
\thispagestyle{fancy}

\section*{Task 1 - Formulating Problems}

\begin{enumerate}
	\item[a)] 
	The states would be the configuration of the chess pieces on the board. 
	These are discrete and finite states, but not every sate can be reached sometimes, depending on the current configuration of pieces.
	The actions are the specific actions which all the pieces in the current state can perform. 
	These are discrete and finite, as well. 
	For rewards, it would be good to punish when a piece is removed and reward actions which lead to check or other movements which may lead to winning.
	\item[b)] 
	As states, we would use the position of the arm. 
	These are therefore discrete and finite.
	This way, the actions would be the movement of the arm, which are finite as well. 
	A positive reward should be given if the robot can pick objects and performs fast, this means taking a long time or dropping objects would be punished with a negative reward. 
	\item[c)] 
	There could be two discrete states, stable and unstable. 
	These indicate whether the drone must be stabilised or not. 
	As actions, the movement of the drone, e.g. left, right, up, down, forward, back could be used. 
	These are finite and can be continuous since every position is possible. 
	A positive reward can be given if the drone flies balanced and a negative one if it needs a lot of time to balance out. 
	\item[d)] 
	
\end{enumerate}

\section*{Task 2 - Value Functions}

\section*{Task 3 - Bruteforce the Policy Space}

\end{document}
